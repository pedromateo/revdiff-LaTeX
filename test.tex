\documentclass[11pt,onecolumn]{article}
\usepackage[T1]{fontenc}


%%%
%%% package for revision mode
\usepackage[revision]{revdiff}
%\usepackage[new]{revdiff}
%\usepackage[clean]{revdiff}

%%%
%%% document starts here
\begin{document} 

\date{\vspace{-5ex}}
\title{{\huge{\bf revdiff}}\\ A \LaTeX ~revision and diff package}
\maketitle


\begin{abstract}
{\bf revdiff} was created to build diff documents with which authors
can ease the work of reviewers by marking changes in the text, adding
tags and adding comments.
\end{abstract}

%%%
\section{Printing built-in legend} 

\rlegend

(This text was printed using {\tt \textbackslash rlegend} command.)

%%%
\section{Modes}

There are currently three modes:
\begin{description}
	\item[revision] prints \textcolor{commentcolor}{comments}, \noindent \raisebox{-1.2ex}{\tikz{\node[text height=1.5ex,text depth=.5ex,scale=.9,fill=tagcolor!50,draw=tagcolor!100,thick,rounded corners] {\tt tags};}}, as well as \textcolor{oldcolor}{old} and \textcolor{newcolor}{new} text revisions in color coded format (as shown above).
	\item[new] hides comments and tags, and displays old text as \textcolor{oldcolor}{[\ldots]} while printing \textcolor{newcolor}{new text in highlighted color}.
	\item[clean] only shows the new text version without any revision markup.
\end{description} 

%%%
\section{Usage example} 

Lorem ipsum dolor sit amet, \rnew{this is new text} consectetur
adipiscing elit, sed do eiusmod tempor incididunt ut labore et dolore
magna aliqua. Ut enim ad minim veniam \rold{this is old text}, quis
nostrud exercitation ullamco laboris nisi ut aliquip ex ea commodo
consequat.

\rcomment{This is a comment}\\ Duis aute irure dolor in reprehenderit
in voluptate velit esse cillum dolore eu fugiat nulla
pariatur. \rchange{This is an}{inline change} Excepteur sint occaecat
cupidatat non proident, sunt in culpa qui officia deserunt mollit anim
id est laborum. \rtag{a tag}



%%%
\section{TODO list}

\begin{itemize} 
\item tag index including links to tags. Index is created by using
  {\tt \textbackslash rtagindex} command.
\end{itemize} 







\end{document}
